\documentclass[10pt,a4paper]{article}
\usepackage[utf8]{inputenc}
\usepackage{amsmath}
\usepackage{amsfonts}
\usepackage{amssymb}
\title{DHCP}
\begin{document}
\maketitle
\section{How does DHCP works ?}
There are 4 steps in the DHCP process:
\begin{itemize}
\item \texttt{DISCOVER}: The client that does not yet have an IP address, broadcasts a series of DHCP discover packets in order to locate the DHCP servers.
\item \texttt{OFFER}: Each DHCP server will respond with an IP address for the client to use.
\item \texttt{REQUEST}: The client requests the use of one of the addresses provided.
\item \texttt{ACK/NACK}: The server acknowledges (ACK) or denies (NACK) the use of the address requested by the user.
\end{itemize}
\section{UCL infrastructure}
There are 2 DHCP servers (dhcp1 and dhcp2).
\end{document}