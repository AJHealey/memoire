% Chapter Template

\chapter{Overview of the UCL Network Infrastructure} % Main chapter title

\label{Chapter2} % Change X to a consecutive number; for referencing this chapter elsewhere, use \ref{ChapterX}

\lhead{Chapter 2. \emph{UCL Infrastructure}} % Change X to a consecutive number; this is for the header on each page - perhaps a shortened title

%----------------------------------------------------------------------------------------
%	SECTION 1
%----------------------------------------------------------------------------------------

\section{Overview of the UCL wireless network}
L'Université Catholique de Louvain est l'une des plus grande université de Belgique. Elle rassemble presque 30.000 étudiants et quelques 10.000 autres membres allant du personnel aux enseignants ainsi qu'aux chercheurs.\\
L'Université Catholique de Louvain possède également plusieurs campus. Le siège central de l'Université est situé dans la ville de Louvain-la-Neuve. Le campus regroupant les sciences de la santé se trouve, quant-à-lui, à Woluwe-Saint-Lambert et plus récemment Tournai, Mons ainsi que Charleroi se sont rajoutés à la liste.\\
Face à une telle envergure, il est vital pour l'UCL de se doter d'une connexion internet et d'un réseau wireless performant capable de délivrer une connectivité sur l'ensemble de ses campus et à l'ensemble des ses étudiants et personnel à tout moment.\\
L'optique dans laquelle l'Université Catholique de Louvain s'est inscrite est de fournir une connectivité en fonction du status de l'utilisateur qui souhaite s'y connecter. Pour ce faire, l'Université dispose de trois réseaux principaux ayant chacun un SSID différent. Ces différents réseaux sont:
\begin{itemize}
	\item \texttt{student.UCLouvain}: Uniquement réservé aux étudiants inscrits à l'Université Catholique de Louvain.
	\item \texttt{UCLouvain-prive}: Uniquement réservé au personnel ainsi qu'aux chercheurs de l'Univeristé.
	\item \texttt{UCLouvain}: Accessible pour les visiteurs invités par l'UCL.
\end{itemize}

L'UCL participe également au projet \texttt{eduroam} qui signifie Education Roaming. Il s'agit d'un service d'accès mondial et sécurisé à l'Internet développé pour la recherche internationale et l'éducation. Le système eduroam est basé sur une infrastrucutre RADIUS qui utilise la technologie 802.1X pour permettre ce roaming inter-institutionnel. Il permet aux utilisateurs visitant une autre institution connectée à eduroam de pouvoir se connecter en utilisant leurs mêmes informations (nom d'utilisateur et mot de passe) que s'ils voulaient se connecter au réseau de leur institution d'origine.\\
Nous retrouvons donc également sur les campus de l'Université un quatrième SSID (\texttt{eduroam}) permettant aux étudiants étrangers de pouvoir se connecter à tout moment sur les sites de l'UCL.

%-----------------------------------
%	SUBSECTION 1
%-----------------------------------
\subsection{Subsection 1}

Nunc posuere quam at lectus tristique eu ultrices augue venenatis. Vestibulum ante ipsum primis in faucibus orci luctus et ultrices posuere cubilia Curae; Aliquam erat volutpat. Vivamus sodales tortor eget quam adipiscing in vulputate ante ullamcorper. Sed eros ante, lacinia et sollicitudin et, aliquam sit amet augue. In hac habitasse platea dictumst.

%-----------------------------------
%	SUBSECTION 2
%-----------------------------------

\subsection{Subsection 2}
Morbi rutrum odio eget arcu adipiscing sodales. Aenean et purus a est pulvinar pellentesque. Cras in elit neque, quis varius elit. Phasellus fringilla, nibh eu tempus venenatis, dolor elit posuere quam, quis adipiscing urna leo nec orci. Sed nec nulla auctor odio aliquet consequat. Ut nec nulla in ante ullamcorper aliquam at sed dolor. Phasellus fermentum magna in augue gravida cursus. Cras sed pretium lorem. Pellentesque eget ornare odio. Proin accumsan, massa viverra cursus pharetra, ipsum nisi lobortis velit, a malesuada dolor lorem eu neque.

%----------------------------------------------------------------------------------------
%	SECTION 2
%----------------------------------------------------------------------------------------

\section{Main Section 2}

Sed ullamcorper quam eu nisl interdum at interdum enim egestas. Aliquam placerat justo sed lectus lobortis ut porta nisl porttitor. Vestibulum mi dolor, lacinia molestie gravida at, tempus vitae ligula. Donec eget quam sapien, in viverra eros. Donec pellentesque justo a massa fringilla non vestibulum metus vestibulum. Vestibulum in orci quis felis tempor lacinia. Vivamus ornare ultrices facilisis. Ut hendrerit volutpat vulputate. Morbi condimentum venenatis augue, id porta ipsum vulputate in. Curabitur luctus tempus justo. Vestibulum risus lectus, adipiscing nec condimentum quis, condimentum nec nisl. Aliquam dictum sagittis velit sed iaculis. Morbi tristique augue sit amet nulla pulvinar id facilisis ligula mollis. Nam elit libero, tincidunt ut aliquam at, molestie in quam. Aenean rhoncus vehicula hendrerit.