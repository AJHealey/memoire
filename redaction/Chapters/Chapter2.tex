% Chapter Template

\chapter{Overview of the working environment} % Main chapter title

\label{Chapter2} % Change X to a consecutive number; for referencing this chapter elsewhere, use \ref{ChapterX}

\lhead{Chapter 2. \emph{Overview of the working environment}} % Change X to a consecutive number; this is for the header on each page - perhaps a shortened title

%----------------------------------------------------------------------------------------
%	SECTION 1
%----------------------------------------------------------------------------------------

\section{Overview of the UCL Internet infrastructure}
L'Université Catholique de Louvain est l'une des plus grande université de Belgique. Elle rassemble presque 30.000 étudiants et quelques 10.000 autres membres allant du personnel aux enseignants ainsi qu'aux chercheurs.\\
L'Université Catholique de Louvain possède également plusieurs campus. Le siège central de l'Université est situé dans la ville de Louvain-la-Neuve. Le campus regroupant les sciences de la santé se trouve, quant-à-lui, à Woluwe-Saint-Lambert et plus récemment Tournai, Mons ainsi que Charleroi se sont rajoutés à la liste.\\
Face à une telle envergure, il est vital pour l'UCL de se doter d'une connexion internet et d'un réseau wireless performant capable de délivrer une connectivité sur l'ensemble de ses campus et à l'ensemble des ses étudiants et personnel à tout moment.\\
L'optique dans laquelle l'Université Catholique de Louvain s'est inscrite est de fournir une connectivité en fonction du status de l'utilisateur qui souhaite s'y connecter. Pour ce faire, l'Université dispose de trois réseaux principaux ayant chacun un SSID différent. Ces différents réseaux sont:
\begin{itemize}
	\item \texttt{student.UCLouvain}: Uniquement réservé aux étudiants inscrits à l'Université Catholique de Louvain.
	\item \texttt{UCLouvain-prive}: Uniquement réservé au personnel ainsi qu'aux chercheurs de l'Univeristé.
	\item \texttt{UCLouvain}: Accessible pour les visiteurs invités par l'UCL.
\end{itemize}

L'UCL participe également au projet \texttt{eduroam} qui signifie Education Roaming. Il s'agit d'un service d'accès mondial et sécurisé à l'Internet développé pour la recherche internationale et l'éducation. Le système eduroam est basé sur une infrastrucutre RADIUS qui utilise la technologie 802.1X pour permettre ce roaming inter-institutionnel. Il permet aux utilisateurs visitant une autre institution connectée à eduroam de pouvoir se connecter en utilisant leurs mêmes informations (nom d'utilisateur et mot de passe) que s'ils voulaient se connecter au réseau de leur institution d'origine.\\
Nous retrouvons donc également sur les campus de l'Université un quatrième SSID (\texttt{eduroam}) permettant aux étudiants étrangers de pouvoir se connecter à tout moment sur les sites de l'UCL.


\section{Hardware infrastructure}


\section{Understanding the passive and active logs}