% Chapter Template

\chapter{Working Environment Overview} % Main chapter title

\label{Chapter2} % Change X to a consecutive number; for referencing this chapter elsewhere, use \ref{ChapterX}

\lhead{Chapter 2. \emph{Working Environment Overview}} % Change X to a consecutive number; this is for the header on each page - perhaps a shortened title

%----------------------------------------------------------------------------------------
%	SECTION 1
%----------------------------------------------------------------------------------------

\section{UCL Internet Infrastructure}
The Catholic University of Louvain (UCL) is one of the biggest universities in Belgium. It gathers almost 30.000 students and about 10.000 other members from staff to teachers and researchers.\\
The university also owns several student campus. The headquarters of the UCL is located in the city of Louvain-la-Neuve. The campus gathering the health sciences is located in Woluwe-Saint-Lambert and more recently the cities of Tournai and Mons as well as Charleroi were added to the list.\\
Faced with such a scale, it is vital for the Catholic Univeristy of Louvain to develop a reliable and efficient Internet connection and wireless network able to deliver a connectivity throughout its campus and all users at all time.\\
The purpose the University enrolled in is to provide an Internet access and a connectivity according to the type of user who wants to connect. To do this, there are 3 main networks at the Catholic University of Louvain, each with a different SSID\\.
The univeristy also participates in the projet \texttt{eduroam} (which stands for education roaming). Eduroam is the secure, world-wide roaming access service developed for the international research and education community\cite{eduroam1}.\\
The eduroam system is a RADIUS-based infrastructure that uses the 802.1X security technology to allow for inter-institutional roaming. It allows the users visiting another institution connected to eduroam to log on to the WLAN using the same credentials the user would use if he were at his home institution\cite{eduroam2}.\\
The Catholic University of Louvain thus has a fourth network available with the SSID \texttt{eduroam} allowing the foreign students to be able to get an Internet connection at any time on the university locations.\\
The available networks at UCL are the following:
\begin{itemize}
	\item \texttt{student.UCLouvain}: Only for the students enrolled for the current year at UCL.
	\item \texttt{UCLouvain}: Only for university staff as well as for the researchers.
	\item \texttt{visiteurs.UCLouvain}: Accessible for guests invited by the university.
	\item \texttt{eduroam}: Education Roaming access.
\end{itemize}



\section{Hardware infrastructure}


\section{Understanding the passive and active logs}