% Chapter Template

\chapter{Working Environment} % Main chapter title

\label{Chapter2} % Change X to a consecutive number; for referencing this chapter elsewhere, use \ref{ChapterX}

\lhead{Chapter 2. \emph{Working Environment}} % Change X to a consecutive number; this is for the header on each page - perhaps a shortened title

%----------------------------------------------------------------------------------------
%	SECTION 1
%----------------------------------------------------------------------------------------

\section{UCL's wireless infrastructure}

\subsection{Overview}
On its Louvain-la-Neuve's campus area, the Catholic University of Louvain offers five different networks, each with a different SSID, that are reserved to certain parts of users according to their category. Those available networks are the following:
\begin{itemize}
	\item[-] \texttt{student.UCLouvain}: Available for the students enrolled at UCL.
	\item[-] \texttt{UCLouvain}: Reserved for university's professors and the researchers.
	\item[-] \texttt{UCLouvain-prive}: Limited access to check the UCL's wireless configuration page or to download required program for Windows XP and Vista.
	\item[-] \texttt{visiteurs.UCLouvain}: Accessible for guests invited by the university.
	\item[-] \texttt{eduroam}: International education roaming access.
\end{itemize}

Security is an important issue and an everyday struggle at the university and several measures are regularly taken by the SRI team in order to keep the integrity and the consistency of the UCL's network preserved. As an example, Microsoft has decided on the 8th of April 2014 to end the Windows XP's support and updates leaving all the remaining laptops running this operating system unprotected to possible external threats\cite{windows}. The problem with that Microsoft's decision is that those outdated computers might become more vulnerable to security risks and viruses. It could become a real problem for the university's network because if a contaminated host connects to the network it can cause serious damages to the overall infrastructure depending on what the virus is programmed to do. This is why the SRI team took the decision to block the wireless access to any device that still uses Windows XP since the 8th of April.

The main way of protecting the network infrastructure from exterior threats that has been installed and configured by the SRI team is the use of the IEEE 802.1X standard providing an authentication mechanism for all the devices that want to connect to a UCL's wireless network.
Basically, every student, professor, researcher or staff members needs a special WiFi username and a password to connect to one of the networks. The username is composed of two parts, on one side the login the user has received from the university's administrative system when he enrolled himself, and on the other side the following part \texttt{@wifi.uclouvain.be}. With that username, the user also needs a password. This password is the same as the password he uses to connect to his personnal virtual office on the university's web site. Without those credentials it is impossible to connect to the UCL's wireless network.

This kind of authentication mechanism is only possible if the infrastructure has several key entities that are going to handle all this security authentication process. Those entities are an authentication server and an authenticator. Typically an authentication server is a host that supports the RADIUS and  EAP protocols. In the case of the UCL's infrastructure there are also two servers that support the LDAP (\textit{Lightweight Directory Access Protocol}) protocol where all the information about the students, professors, researchers and staff members of the UCL are stored. Among those data, we find the users' credentials that ensure a wireless connection authentication and access. The authenticator is the access point that handles the connection request from the user and a Cisco controller that handles all the connection process. The authenticator part is the security guard for the network since the client is not allowed to the protected side of the network until it's identity has been validated and authorized.


\section{Network Topology}
Unlike the Woluwe's campus where an Aruba Networks controler is used, the SRI team has chosen to use with a Cisco wireless LAN controler (WLC) for the university's network in Louvain-la-Neuve.



Here is the representation of the network topology:
\begin{figure}[H]
	\includegraphics[width=.9\linewidth]{Pictures/Chapter2/topology.png}
\end{figure}






\subsection{Design}
Using the network monitoring software InterMapper\cite{intermapper} we see that the UCL network is composed of seven neighborhood routers (CtPythagore, CtHalles, CtLew, CtStevin, CtCarnoy, CtMichotte and CtSH1C). Six of them are present on the Louvain-la-Neuve campus and only CtLew is on the Wolluwé Campus. Those routers task is only routing.\\

Internet access is provided by Belnet via a 10GBit ethernet link directly connected to the CtPythagore. There is also a second 3GBit ethernet link connected to the CtHalles router but this link is never used. It is only a backup link in case of failure of the main one.

The infrastrucutre also has two main servers which are CtTier2 and CtAquarium. Those main servers are datacentres that contain the RADIUS servers as well as the LDAP servers.

Then for each building there is a switch and this switch is directly connected to one of the seven routers. Each of those switches has 48 ports that are connected to the access points inside the concerned building.\\
Concretely, in each building we find ethernet plugs that are connected to what we call concentration points. Those points contain commutators that is connected to the building switch that is connected to one of the main routers.\\
An important point to mention is that the network is not a full mesh.

Here is a simplified representation of the UCL network infrastructure:
\begin{figure}[H]
	\includegraphics[width=.9\linewidth]{Pictures/Chapter2/infrastructure.png}
\end{figure}





\section{Understanding the passive and active logs}