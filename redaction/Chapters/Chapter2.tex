% Chapter Template

\chapter{Working Environment} % Main chapter title

\label{Chapter2} % Change X to a consecutive number; for referencing this chapter elsewhere, use \ref{ChapterX}

\lhead{Chapter 2. \emph{Working Environment}} % Change X to a consecutive number; this is for the header on each page - perhaps a shortened title

%----------------------------------------------------------------------------------------
%	SECTION 1
%----------------------------------------------------------------------------------------

%http://www.cisco.com/web/FR/documents/pdfs/livres_blancs/mobilite_sans_fil/Protocole_LWAPP.pdf

\section{UCL's wireless infrastructure}

On its Louvain-la-Neuve's campus area, the Catholic University of Louvain offers five different networks, each with a different SSID, that are reserved to certain parts of users according to their category. Those available networks are the following:
\begin{itemize}
	\item[-] \texttt{student.UCLouvain}: Available for all the students enrolled at UCL.
	\item[-] \texttt{UCLouvain}: Reserved for university's staff
	\item[-] \texttt{UCLouvain-prive}: Limited access to check the UCL's wireless configuration page or to download required program for Windows XP and Vista.
	\item[-] \texttt{visiteurs.UCLouvain}: Accessible for guests invited by the university.
	\item[-] \texttt{eduroam}: International education roaming access.
\end{itemize}

Network security is an important issue and an everyday struggle at the university and several measures are regularly taken by the SGSI team in order to keep the integrity and the consistency of the UCL's network preserved. As an example, Microsoft has decided on the 8th of April 2014 to end the Windows XP's support and updates leaving all the remaining laptops running this operating system unprotected to possible external threats\cite{windows}. The problem with that Microsoft's decision is that those outdated computers might become more vulnerable to security risks and malwares/viruses. It could become a real problem for the university's network because if a contaminated host is able to connect to the network it can cause serious damages to the overall infrastructure. In order to counter this possible problem in the future, the SGSI team plan to block the wireless access, using snorting programs, to any device that still uses Windows XP at that moment.

The main way of protecting the network infrastructure from exterior threats that has been installed and configured by the SRI team is the use of the \texttt{IEEE 802.1X} standard providing an authentication mechanism for all the devices that want to connect to a UCL's wireless network. This architecture uses an authentication with user ID and password. Basically, every student, professor, researcher or staff members needs an username and a password in order to connect to one of the networks and get access to the Internet inside the UCL's campus. The username is composed of two parts, on one side the login the user has received from the university's administrative system when he enrolled himself, and on the other side the realm (\texttt{@wifi.uclouvain.be}). Along with that username, the user also needs a password. This password is the same as the password he uses to connect to his personal virtual office on the university's web site. Without those credentials it is impossible for him to connect to the UCL's wireless network.

This kind of authentication mechanism is only possible if the infrastructure has several key entities that are going to handle all this security authentication process. Those entities are an \textit{authentication server} and an \textit{authenticator}. Typically an authentication server is a host that supports the \texttt{RADIUS} and  \texttt{EAP} protocols. In the case of the UCL's infrastructure the authenticator is also composed of several database servers, placed behind a load balancer, that support the \texttt{LDAP} (\textit{Lightweight Directory Access Protocol}) protocol where all the information about the students, professors, researchers and staff members of the UCL are stored. Among those stored data, there are all the users' credentials. The authenticator is the access points that handles the connection request from the user and a Cisco controller that handles all the authentication process. This authenticator is the real security guard for the network since the client is not allowed to access to the protected side of the network until it's identity has been validated and authorized.

Let's keep in mind that these controller and access points do not play a role in the real world of the authentication itself. They are called the \textit{Authenticator} because that is where the client asks for authentication but it do not authenticate the client. The device that authenticate the client is the \textit{Authentication Server}.


The following figure represents the architecture at the UCL's with the different components that handles the 802.1x authentication process. 

\begin{figure}[H]
	\includegraphics[width=1\linewidth]{Pictures/Chapter2/802-archi.png}
	\caption{802.1x architecture at the UCL}
\end{figure}



\section{Network components and protocols}
In this section, we give more information and details about all the protocols and components used throughout this thesis.

\subsection{EAP}
\texttt{EAP} (\textit{The Extensible Authentication Protocol}) is an Internet Engineering Task Force (IETF) flexible authentication framework standard defined in the \texttt{RFC3748} \cite{rfc3748}. EAP was designed to work without the IP protocol and provides support only for the transport of authentication protocols. This protocol runs directly over data link layers such as \texttt{Point-to-Point Protocol} (PPP) or \texttt{IEEE 802}. It is used to select a specific authentication mechanism whenever the authenticator requests more information to the supplicant.

The particularity with \texttt{EAP} is that the authentication mechanism is negotiated by the peers (i.e. the supplicants) during the connection authentication phase with the authentication server. The peers negotiate the use of which \texttt{EAP} method to use during the authentication. Once the method has been agreed upon, an \texttt{EAP} conversation consisting of requests and responses messages exchaged between the peer and the authentication server starts.

The \texttt{EAP} architecture consists of three main elements:
\begin{itemize}
	\item \textbf{EAP peer}: The peer is the client that tries to access the protected network and that responds to the authenticator requests.
	\item \textbf{EAP authenticator}: It is the end of the link initiating an \texttt{EAP} authentication. It is the access point or the network access server that requires \texttt{EAP} authentication in order to grant access to the network.
	\item \textbf{Authentication server}: The server that communicates with the \texttt{EAP} peer. It negotiate the \texttt{EAP} method to use and validates the peer's credentials allowing it to access the protected network or not.
\end{itemize}

Since \texttt{EAP} only defines messages formats it has to be encapsulated inside a data link layer transport protocols. Between the peer and the authenticator, the \texttt{EAP} messages has to be encapsulated inside protocols such as \texttt{PPP}, \texttt{PEAP} or \texttt{IEEE 802.1X}. The encapsulation of \texttt{EAP} over \texttt{IEEE 802.1X} is called \texttt{EAPOL} (\textit{EAP over LANs}). Once the authenticator has received a \texttt{EAP} message from the peer, it has to forward it to the authentication server and it does that by encapsulating the message using the \texttt{RADIUS} protocol. Thanks to that, the \texttt{EAP} messages are exchanged between the peer and the authentication server smoothly. Since the information flows between those two entities, the authenticator does not need to support any of the \texttt{EAP} methods. It just has to forward the messages.

The following figure\footnote{http://technet.microsoft.com/en-us/library/bb457039.aspx} shows the \texttt{EAP} infrastructure and the information flow between the three components.

\begin{figure}[H]
	\includegraphics[width=1\linewidth]{Pictures/Chapter2/eap.png}
	\caption{\texttt{EAP} infrastructure and information flow}
\end{figure}

The \texttt{EAP} method requirements for Wireless LANs is defined in the \texttt{RFC4017}\cite{rfc4017}. The methods used in today's wireless LANs includes \texttt{EAP-TLS}, \texttt{EAP-TTLS}, \texttt{PEAP} and \texttt{EAP-SIM}. As mentioned in the RFC, these methods support authentication credentials that include digital certificates, user-names and passwords, secure tokens, and SIM secrets.

\begin{description}
	\item [EAP-TLS]: Defined in \texttt{RFC5216}\cite{rfc5216}, the \texttt{EAP-TLS} authentication protocol includes support for certificate-based mutual authentication and key derivation, utilizing the protected ciphersuite negotiation, mutual authentication and key management capabilities of the \texttt{TLS} protocol (described in \texttt{RFC4346}\cite{rfc4346}). The security with this protocol is provided by the utilization of \texttt{X.509} certificates. Indeed, in the \texttt{TLS} negotiation, the server presents a certificate to the client, and if mutual authentication is requested, the peer presents its certificate to the server.
	\item [EAP-TTLS]: This protocol, defined in the IETF draft\cite{ttls-draft}, extends \texttt{TLS}. It uses \texttt{TLS} to establish a secure connection between a client and server, through which additional information may be exchanged. The difference with \texttt{TTLS} is that one-way authentication in which only the server is authenticated to the client via its certificate is possible. Once the secure tunnel is established between them, the client can send its credentials (i.e. username and password).
	\item [PEAP]: The Protected Extensible Authentication Protocol is defined in the IETF draft\cite{peap-draft}. This protocol provides an encrypted and authenticated tunnel based on \texttt{TLS} that encapsulates \texttt{EAP} authentication mechanisms. \texttt{PEAP} works by chaining multiple \texttt{EAP} mechanisms. Basically it is composed of two parts. First, a \texttt{TLS} session is negotiated with the server authenticating to the client and optionnaly the client to the server. The negotiated key is used to encrypt the rest of the conversation. Second, within that \texttt{TLS} session, zero or more \texttt{EAP} methods are carried out.
\end{description}


\subsection{IEEE 802.1X}
%http://www.blackhat.com/presentations/win-usa-03/bh-win-03-riley-wireless/bh-win-03-riley.pdf
\texttt{802.1X} is a Port-based Network Access Control that is part of the \texttt{IEEE 802.1} group of networking protocols. Basically, it is a way of securing a network access and doing authentication over ports to devices wishing to attach to a LAN or a WLAN. It offers an effective framework for authenticating and controlling user traffic to a protected network.

The \texttt{802.1X} authentication process architecture involves three main components:
\begin{itemize}
	\item[-]\texttt{A supplicant}: The client device that wants to connect to the network (LAN or WLAN).
	\item[-]\texttt{An authenticator}: A network device like an Ethernet switch or an access point that is between the supplicant and the authentication server. It purpose is to interact with the authentication server. It receives \texttt{EAP} messages, encapsulates them into \texttt{RADIUS} messages and forwards them to the authentication server.
	\item[-]\texttt{An authentication server}: It is the key part of that authentication process. This server support receives the authentication request from the authenticator and interacts with him in order to receive more information or credentials for the supplicant. It is the only one that can grant access or not to the protected network to the user.
\end{itemize} 

This protocol is called \textit{Port-based Network Access Control} because the ports of the switches are configured in a certain way. Indeed, before being totally open, they only accept \texttt{EAP} messages. The idea is that each port is divided into two different parts. One part is the \textit{controlled} part of the port and the other is the \textit{uncontrolled} part of the port. The main idea is that all the \texttt{EAP} messages go through this uncontrolled port.

\texttt{802.1X} works either for WLAN or LAN authentication but the authentication flow is not exactly the same.
For WLAN authentication, the first thing the supplicant does is to associate with the access point using standard \texttt{802.11} communication. It then sends its information inside \texttt{EAP} messages to the authenticator through the uncontrolled port. The authenticator encapsulates those messages into \texttt{RADIUS} packets and sends them to the authentication server. Then, after having set up a secure and encrypted tunnel with the supplicant, the authentication server starts communicating with it (to receive its username and password for example). The server then queries its database and finally, if the user is know, it asks the autenticator to open completely the port so that the user can access the requested VLAN.
The following figure represents this authentication mechanism for WLAN.

\begin{figure}[H]
	\begin{center}
	\includegraphics[width=.7\linewidth]{Pictures/Chapter2/802.png}
	\caption{\texttt{802.1X} authentication mechanism for WLAN}
	\end{center}
\end{figure}

For LAN authentication it is a bit different. Indeed, the user directly connects to the switch through Ethernet. The switch then passes all the supplicant information to the authentication server. The authentication server and the switch exchange authentication and then the server queries its database and asks the switch to open the port or not.
We can see that there is a major difference between WLAN and LAN authentication mechanisms. Indeed, in LAN, there are absolutely no communications between the authentication server and the supplicant. All the communications are done only between the switch and the server, while in WLAN, the authentication server directly communicates with the supplicant through a secured and an encrypted tunnel. 



\subsection{RADIUS}
RFC3579, RFC2865\\
TODO

\subsection{WiSM}
Cisco Wireless Services Module 2 (WiSM2)\\
TODO

\subsection{SNMP}

The Simple Network Management Protocol is an application layer protocol that facilitates the exchange of management information between network devices \cite{snmp}. It is part of the TCP/IP protocol suite and it is mainly used by network administrators to get information about devices on the network and the network performances. These information help the administrators to resolve problems on the network or simply to manage it.
A SNMP network has three main components:

\begin{itemize}
	\item \texttt{Network-management system (NMS)}: A NMS is the main component of an SNMP-managed network. It is the management entity that controls the managed devices. It uses the SNMP protocol and can interact with the managed devices to get information using special commands and messages.
	
	\item \texttt{Managed devices}: It is a network device that contains an SNMP agent. They collect and store information to make them available for the network-management systems. Those devices can be routers, servers, switches,etc. They also run the SNMP protocols to be able to respond to the requests made by the NMS.
	
	\item \texttt{Agents}: An agent is the thinking part of a managed device. It is a software module that understands the management information and translates them into a SNMP compatible form.
\end{itemize}

\begin{figure}[H]
\centering
	\includegraphics[width=.7\linewidth]{Pictures/Chapter2/snmp.png}
	\caption{A typical SNMP managed network}
\end{figure}

All the network objects are described and organized hierarchically in a Management Information Base (MIB). There are MIBs for each set of related network entites that can be managed. These MIBs are accessed using a network-management protocol such as SNMP.



\section{WLAN authentication process}
Let's analyze the authentication process mechanism with a WLAN network.
As explained before, with the \texttt{IEEE 802.1X} standard, there are two key parts, the \textit{Authenticator} and the \textit{Authentication Server}. In the case of a WiFi authentication process, a third one can be added to this list and is called the \textit{Supplicant}. The supplicant is simply the wireless client that asks for an authentication in order to have access to the protected network. The authentication process works in several key steps and uses different protocols.

First of all, the supplicant who wants to access the network needs to make a special request to the authenticator. During that step the protocols that are used between the supplicant and the authenticator are a combination of \texttt{802.1X} and \texttt{EAP} protocol (\textit{Extensible Authentication Protocol}). Thanks to \texttt{802.1X} protocol, the authenticator can refuse any access to the network as long as the authentication server has not authenticated the client and accepted to open the access (i.e. open the port). The \texttt{EAP} defines a standard for the messages that are going to be exchanged between the supplicant and the authentication server. It is a  transport protocol for the authentication protocols. Those \texttt{EAP} messages are encapsulated over \texttt{IEEE 802} and this encapsulation is known as \texttt{EAPOL} (for \textit{"EAP over LAN"}. Technically we should say \texttt{EAPOW} for a wireless network but it is only to refer to an \texttt{EAPOL} message that is being sent using \texttt{802.11} wireless network transmission and the standard never mention \texttt{EAPOW}).

\texttt{EAP} provides several authentication methods. The \texttt{EAP} methods that are used within the UCL's WLAN  security management are either \texttt{EAP-TTLS} (\textit{EAP-Tunneled Transport Layer Security}) or \texttt{EAP-PEAP} (\textit{Protected Extensible Authentication Protocol}) chosen by the OS. They both are an authentication protocol that rely on an encrypted and secured tunnel. With the \texttt{TTLS} method, the client is authenticated with his username and password and the server is authenticated with a \texttt{X509} certificate. When the client starts an authentication process, he uses the server certificate to encrypt all the messages he is going to send to the authentication server. In other words, this certificate is used as a key for encrypting the communication between the supplicant and the authentication server. It is also used to authenticate the authentication server (i.e. to verify that the server the client is talking to is really the UCL's server that authenticate the clients).


Second, once the authenticator has received the request from the supplicant, he has to forward it to the authentication server. The protocol used between the authenticator and the authentication server is the \texttt{RADIUS} protocol. All the requests made by the supplicant, in \texttt{802.1X/EAP} messages, are translated into \texttt{RADIUS} and forwarded to the authentication server. This is know as the \texttt{EAP over RADIUS} protocol.The authenticator server then grants the access or not to the client and sends back a response to the authenticator telling it to open the network access to the client.

As seen on the following picture of the UCL's wireless network topology, the supplicant sends (1) its information to the authenticator (composed of the access point and the WiFi controller) in \texttt{EAP} frames. The authenticator then forwards (2) them in \texttt{RADIUS} encapsulated packets to the authentication server. After queries on the \texttt{LDAP} databases (2) and exchanges with the authenticator, the server sends back (3) a packet saying if the supplicant can access or not to the network. If so, the supplicant is authenticated but still needs an IP address to access the Internet. In order to do so, it asks the \texttt{DHCP} server (4). Once it got its address from the server it can fully access the Internet (5).

fully authenticated and can use (4) the protected network.

\begin{figure}[H]
	\includegraphics[width=.9\linewidth]{Pictures/Chapter2/topology2.png}
	\caption{Authentication process on a UCL's WLANs}
\end{figure}


\subsection{student.UCLouvain authentication example}
To illustrate how the whole authentication process works, let's take a closer look to what happens when a student wants to connect to the \texttt{student.UCLouvain} network.
As detailed above, this network is protected by the \texttt{802.1X} authentication protocol. Thus the client first needs to authenticate himself before being able to access the Internet.
As a reminder, the client who wants to connect the network is called the \textit{supplicant}. This supplicant is going to first make a standard \texttt{802.11} association with the authenticator using \texttt{WPA/WPA2}. The authenticator is composed of the access point to which the supplicant is trying to connect and the WLAN controller that is going to handle the whole authentication request. Once this association has been made, the authenticator understands that the supplicant wants to get an access to the protected network and thus a \texttt{802.1X} session between them is started. At this state, only \texttt{802.1X} traffic is allowed (i.e. only \texttt{EAP} messages are accepted during the transmissions).

The first message the supplicant is going to send to the authenticator is an \texttt{EAPOL-Start} frame in order to initiate the authentication process. When the authenticator receives that frame, it sends an \texttt{EAP-Request Identity} frame to the supplicant that answers with an \texttt{EAP-Response Identity} frame containing the supplicant's identity. The authenticator encapsulates this response in a \texttt{RADIUS Access-Request} packet and forwards it to the authenticator server.

Once the authentication server has received that packet it sends a \texttt{RADIUS Access Challenge} back to the authenticator containing an \texttt{EAP Request} specifying the type of \texttt{EAP} method to use (\texttt{TTLS} or \texttt{PEAP}) in order to validate the identity of the supplicant and to specify how the credentials are submitted. The authentication server also sends its certificate during that negotiation. The authenticator encapsulates that \texttt{EAP Request} in an \texttt{EAPOL} frame and sends it to the supplicant.

Since the supplicant has a copy of the server certificate, it can checks if the one received during the negotiation is the same in order to authenticate the server's identity. Once its identity has been proved, the supplicant builds a \texttt{TLS-encrypted} tunnel with the authentication server. It then sends its credential securely inside this channel.

The authentication server validates the username and password of the supplicant and sends back a \texttt{RADIUS Access-Accept} packet to the authenticator that sends an \texttt{EAP-Success} frame to the supplicant. The authenticator opens the port for the supplicant and by doing so, completing the process of authentication and allowing the user to access the Internet after having negotiated a \texttt{WPA/WPA2} key with the authenticator in order to encrypt the further transmissions.

The following figure represents the main steps of the authentication process for the \texttt{student.UCLouvain} network:

\begin{figure}[H]
	\includegraphics[width=.9\linewidth]{Pictures/Chapter2/student.png}
	\caption{student.UCLouvain authentication process}
\end{figure}


\subsection{eduroam authentication example}
Let's take another WLAN authentication process example with the \texttt{eduroam} network which is a bit different than the example seen before.
As explained in \cite{eduroamRadius}, the \texttt{eduroam} project is "\textit{A wolrdwide federation of} \texttt{RADIUS} \textit{ servers facilitating network access for roaming academic affiliates using} \texttt{IEEE 802.1X} \textit{as the vehicle. eduroam's use of} \texttt{802.1X} \textit{in concert with} \texttt{RADIUS} \textit{means the network is built around well understood, established, and easy to manage standards which are often already deployed within the network infrastructure of educational institutions}".

Since \texttt{eduroam} is also using the \texttt{802.1X}, when a client wants to connect to the network, he is not able to pass any traffic other than \texttt{802.1X} until his request is accepted by the authentication server. As for the \texttt{student.UCLouvain} network, the communication between the supplicant and the authenticator also involves \texttt{EAP} conversation.The supplicant sends its \texttt{EAP} messages to the authenticator that forwards them to the authentication server in the form of a \texttt{RADIUS} request. To ensure the protection of the credentials and information sent during the authentication negotiation, eduroam uses either \texttt{TTLS} or \texttt{PEAP} (\textit{Protected Extensible Authentication Protocol}).A \texttt{TLS-encrypted} tunnel is also created between the supplicant and the authentication server.

As a student from another university than the Catholic University of Louvain can access and connect to the \texttt{eduroam} network from inside the Louvain-la-Neuve campus, the local authentication server, in Louvain-la-Neuve, is not the one that is going to handle the authentication request. Indeed, if the user comes from the University of Seville in Spain, for example, its authentication request within the UCL's campus is first handled by the local \texttt{RADIUS} server. This server forwards the request to the Belnet \texttt{RADIUS} server since it does not know the realm used by the spanish student. That Belnet server understands that this student is not from Belgium. It thus forwards the request to the European \texttt{RADIUS} server, Terena. Terena understands that the user comes from Spain so it forwards the request to the spanish national \texttt{RADIUS} server that finally forwards the request to the University of Seville's server. The response takes the same route to get to Louvain-la-Neuve.

The \texttt{RADIUS} protocol supports that forwarding in its proxy mode. To prevent any administrators not responsible for the handling of the authentication request, the \texttt{TLS-encrypted} tunnel is propagated throughout the \texttt{RADIUS} infrastructure. Thanks to that tunnel, the intermediate authenticator does not handle sensitive information during the authentication process.

Here is a representation of the authentication request communications between the supplicant, the authenticator and the authentication server with \texttt{RADIUS} proxying:
\begin{figure}[H]
	\includegraphics[width=1\linewidth]{Pictures/Chapter2/eduroam1.png}
	\caption{eduroam authentication process}
\end{figure}

\section{Infrastructure architecture}
Within the Catholic University of Louvain's network infrastructure architecture, the Internet access is provided by Belnet via a 10Gbit Ethernet link. This link is connected to one of the seven \textit{neighborhood routers} called \texttt{CtPythagore}. There is also a second 1Gbit Ethernet link connected to another neighborhood router called \texttt{CtHalles}. This second link is never used and is, in fact, a backup link in case of a failure of the main one.

The other neighborhood routers are \texttt{CtLew}, \texttt{CtStevin}, \texttt{CtCarnoy}, \texttt{CtMichotte} and \texttt{CtSHI1C}. They are all on the Louvain-la-Neuve campus expect for \texttt{CtLew} that lies on the Woluwe campus in Brussels. Those neighborhood routers are all \texttt{Cisco Catalyst 6509} switches and are the campus core routers. Two of them (\texttt{CtSH1C} and \texttt{CtMichotte}) include a \texttt{Cisco WiSM2} (\textit{Wireless Services Module 2}) controller. The infrastructure also contains two data centers called \texttt{CtTier2} and \texttt{CtAquarium}. Each one of those data center has a load balancer. The two \texttt{DHCP} servers as well as the \texttt{LDAP} servers are located behind those load balancers.

Here is a representation of the UCL's network infrastructure.

\begin{figure}[H]
	\includegraphics[width=1\linewidth]{Pictures/Chapter2/ucl.png}
	\caption{UCL's network infrastructure architecture}
\end{figure}


Each building on the Louvain-la-Neuve campus has a direct connectivity with the network. Inside there are several access points (there are approximatively 300 access points on the Louvain-la-Neuve's campus and most of them are \texttt{Cisco AIR-AP1242AG}. Yet they are starting to be replaced, only in the INGI buildings for this moment, by either \texttt{Cisco Aironet 3600} or \texttt{Cisco Aironet 3700}) that provide an Internet access to the clients. Each access point is connected to a switch \texttt{Cisco Catalyst 2960-48LPS}. This switch only has 48 ports so there are several one of them, in each building, place in a patch room. Those switches are connected to another switch \texttt{Cisco Catalyst 2960G-24TC-L} that has a direct fiber connection with one of the seven neighborhood routers. Thanks to that connection, each building is connected to the UCL network and thus,to the WiFi controller.

The following figure represents a simplified overview of how the buildings are connected to the network.

\begin{figure}[H]
	\includegraphics[width=1\linewidth]{Pictures/Chapter2/building.png}
	\caption{Building infrastructure and network connection}
\end{figure}







