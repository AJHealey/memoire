% Chapter Template

\chapter{Introduction} % Main chapter title

\label{Chapter1} % Change X to a consecutive number; for referencing this chapter elsewhere, use \ref{ChapterX}

\lhead{Chapter 1. \emph{Introduction}} % Change X to a consecutive number; this is for the header on each page - perhaps a shortened title

%\section{Presentation}
%This project constitute our master thesis. The goal is to implement a set of tools to help the network administrators monitoring the wireless infrastructure. To achieve that, we have to proceed in two steps. First, we will have to collect all the information available. The sources of information are completely heterogeneous. They range from simple logs to active monitoring through customized routers. The main difficulty here will be to aggregate all the information in a coherent and efficient way. The amount of data will force us to choose which ones are pertinent and which ones are not. Once the gathering steps is done, the raw data will be available but they will be useless if the user can't understand and use them. So, the second step will be to analyse and present them to the users. We will have to define the profile of the end-user and understand what are his needs. The success of our work will be directly link to the fact if our implementation is helpful or not. If the data collected are correct but we are unable to present them in the right way, our work would be meaningless.

With nearly 30.000 students and 10.000 staff members (including teachers and researchers), the Catholic University of Louvain (UCL) is the largest french speaking university in Belgium. The university has several campuses based in Brussels, Tournai, Mons, Charleroi and Louvain-la-Neuve, the latter being the widest and the one we focus on throughout this study. On this campus, a wireless internet connection is provided by the university and is available for all the students and staff members for free. To be more precise, there are a total of four wireless connections available in Louvain-la-Neuve. Three of them are developed and maintained by the university itself and are reserved either for the students (with the SSID \texttt{"student.UCLouvain"}), for the staff (with the SSID \texttt{"UCLouvain"}) or only to have a limited Internet access in order to consult the UCL's wireless configuration web page or to download required programs  for Windows XP and Vista (with the SSID \texttt{"UCLouvain-prive"}). The fourth network with SSID \texttt{"eduoram"} is for education roaming. Basically it allows students, researchers and staff members from other universities to get a wireless connection within the UCL's campus without being enrolled to the Catholic University of Louvain.

Providing those wireless accesses for such a large community of users and all over the campus is a real struggle for the UCL's SRI team which is in charge of keeping the networks up and running whatever trouble may happen. Indeed, in order to deliver and ensure, at any time, a quality connectivity with high performances for everyone, it is vital to develop and maintain those networks everyday so that they remain reliable and efficient.
 
In this study we focus on analyzing and implementing a monitoring tool for the wireless network infrastructure of the Louvain-la-Neuve campus that will help the SRI team identify, in real-time, the possible problems that might occur on that infrastructure, allowing them to react and be able to fix them as quickly as possible.

\section{State of the art}
Here we present the state of the art of network monitoring and performance analysis tools.\\



\section{Towards a custom WiFi Monitoring Tool}
As for other universities around the world, the Catholic Univeristy of Louvain offers a large wireless network throughout its different campus. This network provides a direct and reliable connection to all of the students, staff, teachers and researchers of the university at all time. The problem with the UCL infrastructure is that it is quite huge and it is always changing. The UCL/SRI team, which is responsible for the effective development of that infrastructure and its connection with the outside world, is always trying to improve the connectivity on the site by adding access point or upgrading the Cisco controllers for instance.\\

Because of that complexity, the management and the efficiency of that network has become quite difficult and buggy. Indeed, the logfiles produced by the controllers are very verbosed which induce an arduous and tricky work of decryption when the team wants to find and trace a problem that occured before on the system. Furthermore, there are more and more users trying to get a connection on the campus (laptops, smartphones,...). This might causes some disturbance on the network that leads to connectivity problems for the direct user.\\

In this thesis, we discuss the implementation of a WiFi monitoring tool that will help the network administrators managing the wireless infrastructure. To achieve that, we proceed in two steps. First of all, we collect all the information that travels over the network. Those information come from heterogeneous sources (from controllers logfiles to active monitoring logs through customized routers). Second, we have to analyze and process that raw data in a way that is understandable and readable for the end users.

\subsection{Data Gathering}
%As said before, the sources of data are quite heterogeneous. To handle that, we'll have to implement a system that can hold and represent all the information in a consistent way.

To work properly, our monitoring tool needs to gather data from the UCL wireless network. These data comes from various places and are quite heterogeneous. Indeed, our system implementation gathers, and stores into our private server, the logfiles containing all the information about the RADIUS, LDAP and DHCP servers as well as the information about the different WiSMs (Wireless Services Modules). Thanks to those logfiles we have a complete overview of the network status and the components at all time.\\

Moreover, we have designed a custom OpenWRT router that authenticate itself on the UCL network to check if there is a problem during the authentication phase. All the information that the router gathers are also stored into our server. This gives active network status information compared to the passive one collected with the different logfiles.


\subsection{Data Analysis}
%The core will be responsible to centralized, analyse and take the action accordingly the information received from the probes. Its actions will mainly depend on the access that it will have the network. Typical action would be to adapt the controller or inform precisely the administrator of the problems detected. Most of the time, there most difficult is not to be aware of the problem but to understand the causes of it.
The next step is the data analysis. In that phase, our system examines all the logfiles of the server and [TODO]




Throughout this thesis we explain what are the main issues encountered today on the university wireless network and how our monitoring tool is helping the network administrators managing this system. In a first chapter, we present the working environment, and more specifically, the UCL Internet infrastructure. We also discuss the different types of logs we use in our tool. In the following chapter, we present all the network components and protocols used inside the network and where connectivity problems can occur. Finally we provide and describe an implementation of our monitoring tool and we deploy it on the network to gather results and feedback.
