% Chapter Template

\chapter{Introduction} % Main chapter title

\label{Chapter1} % Change X to a consecutive number; for referencing this chapter elsewhere, use \ref{ChapterX}

\lhead{Chapter 1. \emph{Introduction}} % Change X to a consecutive number; this is for the header on each page - perhaps a shortened title

\section{Presentation}
This project constitute our master thesis. The goal is to implement a set of tools to help the network administrators monitoring the wireless infrastructure. To achieve that, we have to proceed in two steps. First, we will have to collect all the information available. The sources of information are completely heterogeneous. They range from simple logs to active monitoring through customized routers. The main difficulty here will be to aggregate all the information in a coherent and efficient way. The amount of data will force us to choose which ones are pertinent and which ones are not. Once the gathering steps is done, the raw data will be available but they will be useless if the user can't understand and use them. So, the second step will be to analyse and present them to the users. We will have to define the profile of the end-user and understand what are his needs. The success of our work will be directly link to the fact if our implementation is helpful or not. If the data collected are correct but we are unable to present them in the right way, our work would be meaningless.

\subsection{Data Gathering}
As said before, the sources of data are quite heterogeneous. To handle that, we'll have to implement a system that can hold and represent all the information in a coherent way.

\subsection{Data Analysis}
The core will be responsible to centralized, analyse and take the action accordingly the information received from the probes. Its actions will mainly depend on the access that it will have the network. Typical action would be to adapt the controller or inform precisely the administrator of the problems detected. Most of the time, there most difficult is not to be aware of the problem but to understand the causes of it.

\section{Motivation}

\section{Objectives}