% Chapter Template

\chapter{Introduction} % Main chapter title

\label{Chapter1} % Change X to a consecutive number; for referencing this chapter elsewhere, use \ref{ChapterX}

\lhead{Chapter 1. \emph{Introduction}} % Change X to a consecutive number; this is for the header on each page - perhaps a shortened title

%\section{Presentation}
%This project constitute our master thesis. The goal is to implement a set of tools to help the network administrators monitoring the wireless infrastructure. To achieve that, we have to proceed in two steps. First, we will have to collect all the information available. The sources of information are completely heterogeneous. They range from simple logs to active monitoring through customized routers. The main difficulty here will be to aggregate all the information in a coherent and efficient way. The amount of data will force us to choose which ones are pertinent and which ones are not. Once the gathering steps is done, the raw data will be available but they will be useless if the user can't understand and use them. So, the second step will be to analyse and present them to the users. We will have to define the profile of the end-user and understand what are his needs. The success of our work will be directly link to the fact if our implementation is helpful or not. If the data collected are correct but we are unable to present them in the right way, our work would be meaningless.

<<<<<<< HEAD
With nearly 30.000 students and 10.000 staff members (including teachers and researchers), the Catholic University of Louvain (UCL) is the largest french speaking university in Belgium. The university has several campuses based in Brussels, Tournai, Mons, Charleroi and Louvain-la-Neuve, the latter being the widest and the one we focus on throughout this study. On this campus, a wireless internet connection is provided by the university and is available for all the students and staff members for free. To be more precise, there is a total of four wireless networks available in Louvain-la-Neuve. Three of them are developed and maintained by the university itself and are reserved either for the students (with the SSID \texttt{"student.UCLouvain"}), for the staff (with the SSID \texttt{"UCLouvain"}) or only to have a limited Internet access in order to consult the UCL's wireless configuration web page or to download required programs  for Windows XP and Vista (with the SSID \texttt{"UCLouvain-prive"}). The fourth network with SSID \texttt{"eduroam"} is for education roaming. Basically it allows students, researchers and staff members from other universities to get a wireless connectivity within the UCL's campus without being enrolled to the Catholic University of Louvain.
=======
With nearly 30.000 students and 10.000 staff members (including teachers and researchers), the Catholic University of Louvain (UCL) is the largest french speaking university in Belgium. The university has several campuses based in Brussels, Tournai, Mons, Charleroi and Louvain-la-Neuve, the latter being the widest and the one we focus on throughout this study. On this campus, a wireless internet connection is provided by the university and is available for all the students and staff members for free. To be more precise, there is a total of four wireless networks available in Louvain-la-Neuve. Three of them are developed and maintained by the university itself and are reserved either for the students (with the SSID \texttt{"student.UCLouvain"}), for the staff (with the SSID \texttt{"UCLouvain"}) or only to have a limited Internet access in order to consult the UCL's wireless configuration web page or to download required programs  for Windows XP and Vista (with the SSID \texttt{"UCLouvain-prive"}). The fourth network with SSID \texttt{"eduoram"} is for education roaming. Eduroam is the secure, world-wide roaming access service developed for the international research and education community\cite{eduroam1}. Basically it allows students, researchers and staff members from other universities to get a wireless connectivity within the UCL's campus without being enrolled to the Catholic University of Louvain.
>>>>>>> 8550f8c6dbce76fe0c7794f2fbf34175366b032e

Providing those wireless accesses for such a large community of users and all over the campus is a real struggle for the UCL's SRI (\textit{Infrastructure des réseaux du Système d'information}) team which is in charge of keeping the networks up and running whatever troubles may happen. Indeed, in order to deliver and ensure, at any time, a quality connectivity with high performances for everyone, it is vital to develop and maintain those networks everyday so that they remain reliable and efficient.
 
In this study we focus on analyzing and implementing a monitoring tool for the wireless network infrastructure of the Louvain-la-Neuve campus that will help the SRI team identify, in real-time, the possible problems that might occur on that infrastructure, allowing them to react and be able to fix them as quickly as possible.


\section{State of the art}
Here we present the state of the art of network monitoring and performance analysis tools.\\



\section{Towards a custom WiFi monitoring tool}
From the state of the art described above, we can see that there is room for research in the wireless network monitoring and performance analysis area.\\
Since the UCL infrastructure is rather complex and uses technologies such as Cisco controllers, Cisco WiSMs, RADIUS, LDAP and DHCP servers it can be quite difficult to track down all the processes that are executed during an authentication request issued by a user who wants to connect to the UCL's wireless network and it is even more difficult to track down a problem that has occurred on that network.

Indeed, the log files produced by the controllers are highly verbose inducing an arduous and tricky work of decryption whenever the SRI team wants to trace down a particular network problem. Moreover, another issue the SRI team faces everyday on the Louvain-la-Neuve area is the fact that there are more and more users trying to get a wireless connection on the campus today than before. This can be explained by the fact that nowadays, a significant part of students (as well as staff members) does not only come and attend to courses with their own laptop but also bring smartphones and tablets. The problem those new devices arises is that they can be inadvertently configured to automatically connect to a WiFi network (in our case one of the UCL's wireless networks), whenever one is reachable, even if the user is not currently using those devices to surf the web. These unused connections, added to the one already made by all the laptops, can lead to an overload on an access point resulting into bad performances behaviors and lower signal strengh, since the user's connection request might be redirected to a farther access point, inducing connectivity or network access problems.

%Throughout this thesis, we discuss the implementation of a custom WiFi monitoring tool for the UCL's network infrastructure that will help network administrators from the SRI team managing the wireless network and identifying possible problems in order to fix them and avoid further disturbances for the end users. \\
%To achieve that, our implementation and vision for this monitoring tool is based on three milestones. First of all, we gather on our private server all the information about what is happening and has happened on the wireless network. Once we have collected these two kind of raw data, that we qualify as \textit{active logs} and \textit{passive logs}, we analyze and store them into our custom database aggregating and removing useless data. Finally, we have developed an online platform on which we present the relevant data about the network's state in real time and in a way that is more convenient and understandable for the SRI team users.\\

Our implementation and vision for this monitoring tool is based on three milestones. First of all, we gather on our private server all the information about what is happening and has happened on the wireless network. Once we have collected these two kind of raw data, that we qualify as \textit{active logs} and \textit{passive logs}, we analyze and store them into our custom database aggregating them and removing useless data. Finally, we have developed an online platform on which we present the relevant data about the network's state in real time and in a way that is more convenient and understandable for the SRI team users than the overly verbose log files.

Throughout this thesis, we discuss the implementation of a custom WiFi monitoring tool for the UCL's network infrastructure that will help network administrators from the SRI team managing the wireless network and identifying possible problems in order to fix them and avoid further disturbances for the end users. In a first chapter, we present the working environment and more specifically, the UCL's Internet infrastructure. We also give a further detailled discussion about the data our system needs to gather and analyze in order to work properly. In another chapter, we present all the network key components and protocols that are used inside the university's network and where connectivity problems might occur. Finally we provide and describe an implementation of our monitoring tool and we discuss the gathered results and feedbacks after having deployed it on the Catholic University of Louvain's wireless network.

The following subsections give a quick overview of the two first milestones.


\subsection{Data gathering}
The first important step in our implementation is the network's raw data gathering process. Indeed, to work properly our monitoring tool needs to gather all the information available about what is happening and has happened on the UCL's wireless network.\\
These data come from several heterogenous sources. For the \textit{active logs} (i.e. real time information), we have developed and implemented a C script that runs on an OpenWRT router. This script uses the \texttt{wpa\_supplicant} control interface to make a never ending connection loop to each of the four available SSIDs on the campus in order to make a complete authentication chain and to see if each of those networks is reachable or not. The script inserts every process' event response it receives from wpa\_supplicant into a log file that it then sends after a specified amount of time to our server. The routers equiped with this script are thus able to send real time information of the different networks' status.

The second kind of information we gather are the \textit{passive logs}. For those data we collect log files directly from the Cisco controllers. These files contain all the details about the key elements of the network infrastructure which are the RADIUS, LDAP and DHCP servers as well as information about all the different WiSMs (Wireless Services Modules). As for the active logs, those files are also received and stored into our private server.



%To work properly, our monitoring tool needs to gather data from the UCL wireless network. These data comes from various places and are quite heterogeneous. Indeed, our system implementation gathers, and stores into our private server, the logfiles containing all the information about the RADIUS, LDAP and DHCP servers as well as the information about the different WiSMs (Wireless Services Modules). Thanks to those logfiles we have a complete overview of the network status and the components at all time.

%Moreover, we have designed a custom OpenWRT router that authenticate itself on the UCL network to check if there is a problem during the authentication phase. All the information that the router gathers are also stored into our server. This gives active network status information compared to the passive one collected with the different logfiles.


\subsection{Data analysis}
Once information either from the controllers or the routers is gathered and stored into the server, the analyzing phase begins.\\
We have designed and implemented in Python a parser that goes through all the files we have on the server and that parses all piece of information from them. It then inserts the valuable information into our custom database and rejects all the unecessary and useless one. By doing so, our monitoring tool now has all the UCL wireless infrastructure status and details in hands and we can start aggregating them in order to produce a readable and relevant output that we show on our online platform.\\


%The core will be responsible to centralized, analyse and take the action accordingly the information received from the probes. Its actions will mainly depend on the access that it will have the network. Typical action would be to adapt the controller or inform precisely the administrator of the problems detected. Most of the time, there most difficult is not to be aware of the problem but to understand the causes of it.
%The next step is the data analysis. In that phase, our system examines all the logfiles of the server and [TODO]
