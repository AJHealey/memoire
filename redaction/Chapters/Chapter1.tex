% Chapter Template

\chapter{Introduction} % Main chapter title

\label{Chapter1} % Change X to a consecutive number; for referencing this chapter elsewhere, use \ref{ChapterX}

\lhead{Chapter 1. \emph{Introduction}} % Change X to a consecutive number; this is for the header on each page - perhaps a shortened title

With nearly 30,000 students and 10,000 staff members (including teachers and researchers), the Catholic University of Louvain (UCL) is the largest french speaking university in Belgium. The university has several campuses based in Brussels, Tournai, Mons, Charleroi and Louvain-la-Neuve, the latter being the widest and the one we focus on throughout this study. On this campus, a wireless internet connection is provided by the university and is available for all the students and staff members for free. To be more precise, there is a total of five wireless networks available in Louvain-la-Neuve. Four of them are developed and maintained by the university itself and are reserved either for the students (with the SSID \texttt{"student.UCLouvain"}), for the staff (with the SSID \texttt{"UCLouvain"}), for the visitors (with the SSID \texttt{"visiteurs.UCLouvain"}) or only to have a limited Internet access in order to consult the UCL's wireless configuration web page or to download required programs  for Windows XP and Vista (with the SSID \texttt{"UCLouvain-prive"}). The fifth network with SSID \texttt{"eduroam"} is for education roaming. Eduroam is the secure, world-wide roaming access service developed for the international research and education community\cite{eduroam1}. Basically it allows students, researchers and staff members from other universities to get a wireless connectivity within the UCL's campus without being enrolled to the Catholic University of Louvain.

Providing those wireless accesses for such a large community of users and all over the campus is a real struggle for the UCL's SRI (\textit{Infrastructure des réseaux du Système d'information}) team which is in charge of keeping the networks up and running whatever troubles may happen. Indeed, in order to deliver and ensure, at any time, a quality connectivity with high performances to everyone, it is vital to develop and maintain those networks everyday so that they remain reliable and efficient.
 
In this study we focus on analyzing and implementing a monitoring tool for the wireless network infrastructure of the Louvain-la-Neuve campus that will help the SRI team identify, in real-time, the possible problems that might occur on that infrastructure, allowing them to react and be able to fix them as quickly as possible.


\section{State of the art}
Here we present the state of the art of network monitoring and performance analysis tools.\\
There are several approaches to this subject.

\textbf{A highly scalable monitoring tool}. Machaka \textit{et al.} present in \cite{article1} the development and the testing of a monitoring tool for a large scale Wi-Fi network in order to ensure optimal performance of that network. In a large Wi-Fi network, it is quite difficult to monitor and manage the network without using monitoring tools that don't have numerical and graphical representation. In this paper, the testbed they work on is composed of more than 400 hotspots with more than 615 gateway devices. The metrics they are interested in are uptime and downtime, load average and radio noise and channel. 
They used an already existing monitoring tool, \texttt{SEQUIN}, developed for MPLS-based network and they added three monitoring components to it which are a data collection component, a storage and retrieval component and a visualization component. The data collection process is done using Simple Network Management Protocol (SNMP) and Syslog protocol. The dynamic information consists of polled SNMP data and information computed with this data and the static information includes system configuration, network element configuration, network topology, and monitoring agent information. All the data is stored and summarized using data mining techniques into a database. Finally, the performance of the network is displayed to the network administrator using Google Maps for the overall graphical performance of the network and Google Charts for a numerical representation of those performances.  In this thesis, we reuse the idea of these three components and we add them to our own implementation.

\textbf{A new mechanism for network monitoring and shielding in wireless LAN}. Cheng \textit{et al.} present in \cite{article2} an analyzis of the possible pitfalls of wireless monitoring and an implementation of a multithreaded system running on Maemo operating system, a software platform developed by Nokia, to monitor the wireless network and to solve the security problems for mobile users. Their vision and implementation of a wireless monitoring system involve four modules, namely the Data Sniffer Module, the Protocol Analysis Module, the Data Buffer Module and the User Interface Module. In their system's first step, they use sniffers to capture network packes and observer traffic characteristics. The Protocol Analysis module is then started to analyze those gathered data and the result is stored in the Data Buffer Module. Finally, those results are shown to the users using histogram, plain text, pie charts, grids and so on. This simple yet efficient system architecture is reused in this thesis. We also have divided our system in key parts such as the data gathering module, the data analyzis module and the user interface module. Multithreading technology is also reused in our own implementation.

\textbf{From MAP to DIST: The evolution of a large-scale WLAN monitoring system}. Tan \textit{et al.} present in \citep{article3} a working implementation of a large-scale WLAN monitoring system at Dartmouth College. This system is called DIST (for Dartmouth Internet Security Testbed) and is based on their previous project, MAP. This College was among the first to provide a large-scale WLAN connectivity. There are more than 1300 Aruba AP70 access points installed on the campus populated by over 6,000 students and 2,500 faculty and staff. Their system is based on a client/server mode. They use sniffers flashed with OpenWrt Linux capturing and forwarding traffic to their servers via the campus network. The team also added some security in their monitoring tool. Indeed, the security is quite important because they collect data containing sensitive information about the users of the network. Therefore they have decided to encrypt the data exchanged between the sniffers and the servers using a fast stream cipher with a 128-IV. They also integrated data authenticity and integrity with HMAC-SHA256 to achieve a higher level of security. In our own monitoring tool we have reused the idea of client/server mode. Indeed, among our gathering modules we also have sniffers flashed with OpenWrt Linux that exchange logs with our server. In order to ensure security in our system we also use data encryption and data authenticity and integrity.



\section{Towards a custom WiFi monitoring tool}
From the state of the art described above, we can see that there is room for research in the wireless network monitoring and performance analysis area.\\
Since the UCL infrastructure is rather complex and uses technologies such as Cisco controllers, Cisco WiSMs, RADIUS, LDAP and DHCP servers it can be quite difficult to track down all the processes that are executed during an authentication request issued by a user who wants to connect to the UCL's wireless network and it is even more difficult to track down a problem that has occurred on that network.

Indeed, the log files produced by the controllers are highly verbose inducing an arduous and tricky work of decryption whenever the SRI team wants to trace down a particular network problem. Moreover, another issue the SRI team faces everyday on the Louvain-la-Neuve area is the fact that there are more and more users trying to get a wireless connection on the campus today than before. This can be explained by the fact that nowadays, a significant part of students (as well as staff members) does not only come and attend to courses with their own laptop but also bring smartphones and tablets. The problem those new devices arises is that they can be inadvertently configured to automatically connect to a WiFi network (in our case one of the UCL's wireless networks), whenever one is reachable, even if the user is not currently using those devices to surf the web. These unused connections, added to the one already made by all the laptops, can lead to an overload on an access point resulting into bad performances behaviors and lower signal strengh, since the user's connection request might be redirected to a farther access point, inducing connectivity or network access problems.


Our implementation and vision for this monitoring tool is based on three milestones. First of all, we gather on our private server all the information about what is happening and has happened on the wireless network. Once we have collected these two kind of raw data, that we qualify as \textit{active logs} and \textit{passive logs}, we analyze and store them into our custom database aggregating them and removing useless data. Finally, we have developed an online platform on which we present the relevant data about the network's state in real time and in a way that is more convenient and understandable for the SRI team users than the overly verbose log files.

Throughout this thesis, we discuss the implementation of a custom WiFi monitoring tool for the UCL's network infrastructure that will help network administrators from the SRI team managing the wireless network and identifying possible problems in order to fix them and avoid further disturbances for the end users. In a first chapter, we present the working environment and more specifically, the UCL's Internet infrastructure. We also give a further detailled discussion about the data our system needs to gather and analyze in order to work properly. In another chapter, we present all the network key components and protocols that are used inside the university's network and where connectivity problems might occur. Finally we provide and describe an implementation of our monitoring tool and we discuss the gathered results and feedbacks after having deployed it on the Catholic University of Louvain's wireless network.

The following subsections give a quick overview of the two first milestones.


\subsection{Data gathering}
The first important step in our implementation is the network's raw data gathering process. Indeed, to work properly our monitoring tool needs to gather all the information available about what is happening and has happened on the UCL's wireless network.\\
These data come from several heterogenous sources. For the \textit{active logs} (i.e. real time information), we have developed and implemented a C program that runs on an OpenWRT router. This script uses the \texttt{wpa\_supplicant} control interface to execute a connection loop to each of the four available SSIDs on the campus in order to make a complete authentication chain and to see if each of those networks is reachable or not. The script inserts every process' event response it receives from wpa\_supplicant into a log file that it then sends after a specified amount of time to our server. The routers equiped with this script are thus able to send real time information of the different networks' status.

The second kind of information we gather are the \textit{passive logs}. For those data we collect log files directly from the Cisco controllers. These files contain all the details about the key elements of the network infrastructure which are the \texttt{RADIUS}, \texttt{LDAP} and \texttt{DHCP} servers as well as information about the \texttt{WiSM} (\textit{Wireless Services Modules}). As for the active logs, those files are also received and stored into our private server.



\subsection{Data analysis}
Once information either from the controllers or the routers is gathered and stored into the server, the analyzing phase begins.\\
We have designed and implemented in Python a parser that goes through all the files we have on the server and that parses all piece of information from them. It then inserts the valuable information into our custom database and rejects all the unecessary and useless one. By doing so, our monitoring tool now has all the UCL wireless infrastructure status and details in hands and we can start aggregating them in order to produce a readable and relevant output that we show on our online platform.\\


%The core will be responsible to centralized, analyse and take the action accordingly the information received from the probes. Its actions will mainly depend on the access that it will have the network. Typical action would be to adapt the controller or inform precisely the administrator of the problems detected. Most of the time, there most difficult is not to be aware of the problem but to understand the causes of it.
%The next step is the data analysis. In that phase, our system examines all the logfiles of the server and [TODO]
