% Chapter Template

\chapter{Conclusion and Future Talks} % Main chapter title

\label{Chapter 7} % Change X to a consecutive number; for referencing this chapter elsewhere, use \ref{ChapterX}

\lhead{Chapter 7. \emph{Conclusion and Future Talks}} % Change X to a consecutive number; this is for the header on each page - perhaps a shortened title

%----------------------------------------------------------------------------------------
%	SECTION 1
%----------------------------------------------------------------------------------------
\section{Prototype}
This wireless monitoring application is still a \emph{prototype} version and a lot of improvements can be done. About one year ago, we have started from scratch with few knowledge of the UCL's network infrastructure and used network protocols. Today, our application that is able to analyse and monitor the network and understands the data it deals with. 

Like most of new projects, we took a lot of time working on the design of the architecture. Notice that, modularity and maintainability were our two main concerns throughout this project. What we have created here is a solid foundation that allow to adapt the system to lot of context. Our application was implemented to be as dynamic as possible. For example, the \emph{Access Point} and their \emph{interfaces} are detected automatically.


