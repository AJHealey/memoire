% Chapter Template

\chapter{Conclusion and Future Talks} % Main chapter title

\label{Chapter7} % Change X to a consecutive number; for referencing this chapter elsewhere, use \ref{ChapterX}

\lhead{Chapter 7. \emph{Conclusion and Future Talks}} % Change X to a consecutive number; this is for the header on each page - perhaps a shortened title

%----------------------------------------------------------------------------------------
%	SECTION 1
%----------------------------------------------------------------------------------------
\section{Beta Version}
As we conclude this project, we wanted to remind you that this is only a \emph{beta} version and a lot of improvements can be done. About one year ago, we have started from scratch with few knowledge of the UCL infrastructure and how most of the used protocols work. Today, we have an application that is able to analyse and has a basic understanding of the data it deals with. Like most of new project, we used a lot of time in the design of the architecture. We think that it is the most important when we want to ensure a possible future to our system. We hope that our thesis will be continued by future students and we wonder that it become a complete monitoring tool. Modularity and maintainability were our two main concerns.

\section{Portability}


\section{Ethical Considerations}
% introduite dans le ch5->logparser->radius
Several time we needed to limit the possibility of the application. The goal of the system is to offer a global view to the network administrators and giving them the most possible information. So, we have started to implement basic features to get the simplest pieces of data. Once that was done, we have listed the possible information that could be generated by crossing the data and implement the ad hoc methods. Quickly, it appears that the capabilities of the application could become sensitive. We have taken care of generating a modular infrastructure that make easy the creation of new analysis but some of them can be seen as unethical. For example, the \emph{radius logs} make possible to trace an user through the day and represent his position on a map. Such features could seem useful to detect and to diagnostic precisely a problem but we have chosen to not implement them. The question was "where the limit is?" and to answer it we simply used the requirements of the network administrators. We have translated them to a \emph{goal model} and for each goal we tried to find the proper answer. It appears that most of them was solvable by aggregating the data. By proceeding in that way, we completely un-personalized the information. 
Nevertheless, we can't bride completely the system and ensure a complete ethical utilization of it. We think it's really important to keep in mind that, even if the purpose is to offer a better service, the privacy is an important concern when we choose to extends the project.