% Chapter Template

\chapter{Conclusion} % Main chapter title

\label{Chapter 7} % Change X to a consecutive number; for referencing this chapter elsewhere, use \ref{ChapterX}

\lhead{Chapter 7. \emph{Conclusion}} % Change X to a consecutive number; this is for the header on each page - perhaps a shortened title

%----------------------------------------------------------------------------------------
%	SECTION 1
%----------------------------------------------------------------------------------------
We provided an active and passive wireless monitoring application designed for the Catholic University of Louvain's wireless network infrastructure. This application is divided into two key parts, one dealing with the active monitoring data gathering process, and another dealing with the passive data gathering one. We focused on a system that is able to collect information about the current state of the network, display its performances using graphical and numerical representations, detect any possible anomalies and notify the administrators through the web platform we have developed.

We discussed about the working environment we had to deal with, the architecture of the wireless infrastructure as well as all the network components and the protocols used within that infrastructure. We also described and detailed some practical examples about the \texttt{IEEE 802.1X} authentication process used inside the UCL's wireless network.

The architecture design of our application and its implementation was provided as well as deployment tests to verify the correctness and scalability of our wireless monitoring tool. Details about the portability of our application and the procedure to follow in order to add extensions were discussed.  Finally we presented all the results we have collected thanks to all the metrics monitored by our application and discussed about the investigations and conclusions we made about the issues we spotted during the deployment phase.

Our work has several assets. Modularity, simplicity and portability were the three main concerns throughout this project. Our application has been implemented in order to be as dynamic as possible and to be easily extended to implement additional features. Indeed, we discussed how easy it is to add an \texttt{SNMP} object to monitor in the passive monitoring process, or to add a new wireless network within the active monitoring program. Moreover, our tool offers a unique dual view of the network's status through time. One view comes from the components that directly manage the network and the other comes from the devices that actually use that same network. Gathering data from both sides, aggregating and plotting that information helped us getting a better understanding of how the network behaves and where possible issues might occur.

There are also some limitations. In addition to those directly imposed by the router's specifications, that are beyond our control, we do not actually make high level monitoring analysis for the controller, the \texttt{RADIUS} and the \texttt{DHCP} servers. Indeed, during the development phase we did not have a direct access on those components in order to automatically collect their log files. We had to work with old samples. Nevertheless we were able to perform interesting analysis such as the ratio of the different categories of messages present in the controller's log, or a graph comparing the quantity of \texttt{DHCP} packets sent overtime, and even the \texttt{RADIUS} authentication success and fail rate. With an access on these components we would have performed live monitoring of those log files allowing us to use several other interesting metrics in order to give a more precise overview of their current state.

For the future work, we can address the servers access issue stated above. We can also go one step further and use another programming language than \texttt{Python}, such as \texttt{Java} or \texttt{C}, for the server's processes implementation. Indeed, \texttt{Python} is a great language for prototypes but it also has some bad computational performances and memory management issues. More, the database design can be enhanced in order to store more wisely the collected data and to perform faster analysis. Online analytical processing (\texttt{OLAP}) and multidimensional databases could be an interesting approach. 

We are confident that our work in this thesis will help the UCL's network administrators getting a better overview of the current network state and the possible issues in order to provide a fast response.


%\section{Prototype}
%This wireless monitoring application is still a \emph{prototype} version and a lot of improvements can be done. About one year ago, we have started from scratch with few knowledge of the UCL's network infrastructure and used network protocols. Today, our application that is able to analyse and monitor the network and understands the data it deals with. 

%Like most of new projects, we took a lot of time working on the design of the architecture. Notice that, modularity and maintainability were our two main concerns throughout this project. What we have created here is a solid foundation that allow to adapt the system to lot of context. Our application was implemented to be as dynamic as possible. For example, the \emph{Access Point} and their \emph{interfaces} are detected automatically.


%\section{Ethical Considerations}
% introduite dans le ch5->logparser->radius
%Several time we needed to limit the possibility of the application. The goal of the system is to offer a global view to the network administrators and give them the more information possible. We have started to implement basic features to get the simplest pieces of data. Once that was done, we have listed the possible information that could be generated by crossing the data and implement the ad hoc methods. Quickly, it appears that the capabilities of the application could become sensitive. We have taken care of generating a modular infrastructure that make easy the creation of new analysis but some of them can be seen as unethical. For example, the \emph{radius logs} make possible to trace an user through the day and represent his position on a map. Such features could seem useful to detect and to diagnostic precisely a problem but we have chosen to not implement them. The question was "where the limit is?" and to answer it we simply used the requirements of the network administrators. We have translated them to a \emph{goal model} and for each goal we tried to find the proper answer. It appears that most of them was solvable by aggregating the data. By proceeding in that way, we completely un-personalized the information. 
%Nevertheless, we can't bride completely the system and ensure a complete ethical utilization of it. We think it's really important to keep in mind that, even if the purpose is to offer a better service, the privacy is an important concern when we choose to extends the project.

%\section{Conclusion}
