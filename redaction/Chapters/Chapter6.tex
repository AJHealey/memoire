% Chapter Template

\chapter{Analysis and Results} % Main chapter title

\label{Chapter6} % Change X to a consecutive number; for referencing this chapter elsewhere, use \ref{ChapterX}

\lhead{Chapter 6. \emph{Analysis and Results}} % Change X to a consecutive number; this is for the header on each page - perhaps a shortened title

\section{Analysis}
The final purpose of our system is to provide analysis on the main issues present on the infrastructure. The way we processes here is quite straightforward.

\begin{enumerate}
\item Putting together the related information
\item Spotting the anomalies
\item Proposing adequate and possible cures
\end{enumerate}

Like any system, we had to define domains on which the tool will be able to work. One of the first steps of our thesis was to determine what were the main issues. It's here that we will use those. Each of these problems will be a domain in which we will define what are the relevant data, how they can help us and how to react after having analysed them. The purpose of this chapter is to present you these domains and the methodology used inside them. Each of them have their particular needs and their adequate treatments.

\section{Wifi}
\subsection{Users}
This section of the application offers some aggregated statistics about the users of the \emph{wifi network}. The main purpose is to get some analyse of the actual utilization of the wireless infrastructure. It's more a set of indicators to help the administrators to get a global view of his network. 
\subsubsection*{Wifi Protocol}
This graph display the repartition of the users among the available \emph{802.11} standards (i.e. 802.11a, b, g, n). The reason of this analyse come from a earlier policy enforced on the UCL network that forbid to use the \emph{802.11b} protocol. Users connecting with such low bandwidth standard force the \emph{access point} to transmitting longer and can in consequence prevent the others to achieve higher receiving speed. Our analyse was created to help the administrators in future considerations in the same domain. By profiling the habits of the users, we can define the impact of such decision. As the quality of the service is one of the main concern, deactivating the utilization of a standard used by most of the devices could have heavy consequences.
The statistics are based by default on the connections of the last three months. It's to ensure that devices no longer using the network doesn't affect the result and keep the analyse as up to date as possible.
\paragraph*{Source} The data used come from \emph{SNMP}. At each cycle gathering the information of the \emph{devices} associated with the \emph{access points}, we get the information thanks to the value available on the controller. All the descriptions of the OIB come from the Cisco OID Browser \footnote{http://tools.cisco.com/Support/SNMP/do/BrowseOID.do}.

\begin{tabular}{|r l|}
\hline
\textbf{Object} & \texttt{bsnMobileStationProtocol} \\
\textbf{Description} & \parbox{11cm}{The 802.11 protocol type of the client. The protocol is mobile when this client detail is seen on the anchor i.e it's mobility status is anchor.} \\
\textbf{OID} & 1.3.6.1.4.1.14179.2.1.4.1.25 \\
\textbf{MIB} & AIRESPACE-WIRELESS-MIB \\
\hline
\end{tabular}

\subsubsection*{SSID Utilization}
Here, we count the number of users by SSID. As before, the goal is to provide a profile of the users to the administrators. Such measures can help to adapt the policies related to each \emph{VLAN} and defining the load of each of them. It could help to diagnostic a useless VLAN or in the opposite, allow to detect an overloaded that could be cause by an improper 
use of the network. The supposed utilization and the actual one can be really different and such indicators can help to adapt the configurations.
\paragraph*{Source} As before, we mainly aggregate the data available on the controller.

\begin{tabular}{|r l|}
\hline
\textbf{Object} & \texttt{bsnMobileStationSsid} \\
\textbf{Description} & \parbox{11cm}{The SSID Advertised by Mobile Station.} \\
\textbf{OID} & 1.3.6.1.4.1.14179.2.1.4.1.7 \\
\textbf{MIB} & AIRESPACE-WIRELESS-MIB \\
\hline
\end{tabular}

\subsection{Access Point}
This section group several analyse related to the \emph{access points}. In an network with several hundreds of \emph{access point}, it can be difficult to diagnostic and monitor each of them. By centralising and aggregating the data of all of them, we allow the network managers to save time by automatising lot of computation. In this part of the application, we have access to the monitoring of each access point currently associated with the controller.
\subsubsection{Population}

\subsubsection*{Load}
A fundamentals information about an \emph{access point} is the load. To achieve that we monitor each spot continuously and record their state several time per hour. By analysing these data, we are able to plot graph representing the load throughout the days. To achieve that, we need two information: the quantity of data transiting in the link and the maximum speed of the link. Concretely, each \emph{AP} own two byte counter and these are incremented each time a byte is received or send. The other information required is the link speed connecting the spot. This information alone can be useful to diagnostic bandwidth issues. In a big network, it can be difficult to keep track of which links were update and which ones were not. By collecting these data, we found that some \emph{access point} were still connected to \emph{10 Mbits} links. In this case, the main cause was that the concerned devices were not manage by the same service and in consequence were not update with the others. This is the proof that even such simple data can help to diagnostic some inconsistency in the network.
\paragraph*{Source} Here again, we use the data available on the controller. The first data collected are the bytes counter of each \emph{access points}.

\begin{tabular}{|r l|}
\hline
\textbf{Object} & \texttt{cLApEthernetIfRxTotalBytes} \\
\textbf{Description} & \parbox{11cm}{This object represents total number of bytes in the error-free packets received on the interface.} \\
\textbf{OID} & 1.3.6.1.4.1.9.9.513.1.2.2.1.13 \\
\textbf{MIB} & CISCO-LWAPP-AP-MIB \\
\hline
\end{tabular}

\begin{tabular}{|r l|}
\hline
\textbf{Object} & \texttt{cLApEthernetIfTxTotalBytes} \\
\textbf{Description} & \parbox{11cm}{This object represents total number of bytes in the error-free packets transmitted on the interface.} \\
\textbf{OID} & 1.3.6.1.4.1.9.9.513.1.2.2.1.14 \\
\textbf{MIB} & CISCO-LWAPP-AP-MIB \\
\hline
\end{tabular}

The next step is to obtain the speed of each link connected to an \emph{access point}. This time again we use the \emph{SNMP} protocol to get the information from the controller.

\begin{tabular}{|r l|}
\hline
\textbf{Object} & \texttt{cLApEthernetIfLinkSpeed} \\
\textbf{Description} & \parbox{11cm}{Speed of the interface in units of 1,000,000 bits per second.} \\
\textbf{OID} & 1.3.6.1.4.1.9.9.513.1.2.2.1.11 \\
\textbf{MIB} & CISCO-LWAPP-AP-MIB \\
\hline
\end{tabular}

\subsubsection*{Interfaces}
Each \emph{access point} has two wifi interfaces. The first one emits at \emph{5Ghz} and the second on \emph{2,4Ghz}. For each interface, the controller provides a lot of information completing the previous ones. The first that we gather is the number of user currently associated to the interface. we compare these data with the quantity of users with a \emph{poor Signal-Noise Ratio} (SNR). This analyse allow to diagnostic the efficiency of the access point. Even if it can be considered normal to have a couple of users in this category, if the proportion is constantly high, it can be an indicator that a supplementary \emph{access point} may be useful or that the location of the spot is problematic. The problem have to be put in perspective with the statistics of the number of users and the RF environment (i.e \emph{Rogue Access Point} or physical walls) which is outside the scope of this system. 
Finally, we complete this analyse with a monitoring of the \emph{channel utilization}. A high utilization of the channel results in poor connection quality for the user and could indicates a poor RF condition.

\paragraph*{Source} The controller keeps entries for each interface of each access point. We gather and cross all these data to generate some logs about each \emph{access point} and all the related \emph{interfaces}.

\begin{tabular}{|r l|}
\hline
\textbf{Object} & \texttt{bsnAPIfLoadNumOfClients} \\
\textbf{Description} & \parbox{11cm}{This is the number of clients attached to this Airespace AP at the last measurement interval.} \\
\textbf{OID} & 1.3.6.1.4.1.14179.2.2.13.1.4 \\
\textbf{MIB} & AIRESPACE-WIRELESS-MIB \\
\hline
\end{tabular}

\begin{tabular}{|r l|}
\hline
\textbf{Object} & \texttt{bsnAPIfPoorSNRClients} \\
\textbf{Description} & \parbox{11cm}{This is the number of clients with poor SNR attached to this Airespace AP at the last measurement interval.} \\
\textbf{OID} & 1.3.6.1.4.1.14179.2.2.13.1.24 \\
\textbf{MIB} & AIRESPACE-WIRELESS-MIB \\
\hline
\end{tabular}

\begin{tabular}{|r l|}
\hline
\textbf{Object} & \texttt{bsnAPIfLoadChannelUtilization} \\
\textbf{Description} & \parbox{11cm}{Channel Utilization.} \\
\textbf{OID} & 1.3.6.1.4.1.14179.2.2.13.1.3 \\
\textbf{MIB} & AIRESPACE-WIRELESS-MIB \\
\hline
\end{tabular}



\subsection{Rogue Access Points}

\subsection{Simulation of an user}
It's here that our probing devices will be useful. As they simulate the connection of a lambda user, we can compute the time of each a connection steps and approximate the quality perceived by the users.
\subsubsection*{Choice of the Access Point}
The first thing we have observed is that the device didn't connect to the best \emph{access point} (i.e. the one with the best signal). We can observe in the logs generated by the device that most of the times, it tries to associated with several access points without success. It lists all the near spots broadcasting the desired \emph{Service Set ID} (SSID) from the one with the strongest signal to the one with the poorest and succeed the association only on the \emph{fourth} or \emph{fifth}.

\section{Controller}

\subsection{Components Activity}

\section{DHCP}
These analyses aim to detect \emph{DHCP} related issues. The allocation  of IP addresses is a fundamentals component in our network. Being able to detect when problems appears and advertise them could help to make a quicker and more efficient response.

\subsection{Leases}
The main problem with a DHCP is the allocation of ranges of addresses. It's always difficult to determine the quantity of required IP by VLAN. Moreover, when such problems arise, it can be hard to be aware of them. Most of the time, users don't report the problem and when they do, their descriptions state only that they can't connect which is not very useful. Being able to automatically detect a lack of IP addresses can be very time saving for any administrators. Even is such information are available in the \emph{DHCP} logs and that the servers could send warnings, centralising these information aside data related to all the other components of the infrastructure can lead to more powerful analysis. 

\subsection{Crash}


\section{Radius}

\subsection{Authentication Success Rate}




