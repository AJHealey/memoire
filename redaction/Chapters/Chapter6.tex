% Chapter Template

\chapter{Analysis and Results - Active Monitoring} % Main chapter title

\label{Chapter6} % Change X to a consecutive number; for referencing this chapter elsewhere, use \ref{ChapterX}

\lhead{Chapter 6. \emph{Analysis and Results -  Part 2}} % Change X to a consecutive number; this is for the header on each page - perhaps a shortened title

%----------------------------------------------------------------------------------------
%	SECTION 1
%----------------------------------------------------------------------------------------

%Trying to associate -> rejected some time -> status code = 12 ?
%Connecté et puis disconection abrupte -> code = 3 ?

<<<<<<< HEAD
%Limitations ?
%OpenSSL
%Aruba AP70 was originally designed to be a commercial AP instead of a wireless AM, its processing capability is limited: 266-MH MIPS 4Kc CPU, 28-MB RAM, and 8-MB flash memory storage
=======
\section{Simulation of a user}
We use our \texttt{OpenWrt} routers as probes that simulate the behavior of a lambda user that connects to the university network. In order to do that, we have implemented a \texttt{C} program using \texttt{wpa\_supplicant} and its control interface. This program, launched by the router at boot, tries to establish a connection with each one of the five \texttt{VLANs} of the UCL (i.e. \texttt{student.UCLouvain}, \texttt{UCLouvain}, \texttt{visiteurs.UCLouvain}, \texttt{UCLouvain-prive} and \texttt{eduroam}). During that connection process we compute the time taken by \texttt{wpa\_supplicant} to get a connection and the time taken by \texttt{udhcpc} to get an IP address. Thanks to those information we can approximate the overall quality perceived by the normal users when they want to get an access to the Internet inside the UCL. Beside that, our program also checks if a selection of the most used services by a user are reachable with each of the \texttt{VLANs}. We have decided to check the reachability of the following services:
\begin{itemize}
	\item[-] \texttt{www.google.be}
	\item[-] \texttt{www.smtp.gmail.com}
	\item[-] \texttt{www.github.com}
	\item[-] \texttt{www.uclouvain.be}
	\item[-] \texttt{www.icampus.uclouvain.be}
\end{itemize}

We add these additional information to the log file our program creates and sends to our server for parsing and analysis.

\subsection{Choice of the Access Point}
The first thing our program does when it starts its connection loop is to perform a scan of the networks available where the probe is placed and to write the results inside the log file. Among the results and for each network we have decided to keep the network \texttt{BSSID}, the \texttt{Signal Strength} of the network and its \texttt{SSID}. Those data are interesting for our analysis. Indeed, when the router tries to get a connection, it first tries to associate with an Access Point. But there could be several APs in the area and he might be rejected from one or several of them. If its association request is rejected by one AP, \texttt{wpa\_supplicant} will try to associate with another one, and so on until the request is accepted. Our program gathers all the \texttt{BSSID} of the APs it had an association request with. 
Comparing the scan results for each AP and the APs tried and the one it has a connection with is quite interesting. Indeed, after having performed several tests, we have observed that the device does not always connect to the best AP (i.e. the one with the best signal strength).

TODO
 %https://chromium-review.googlesource.com/#/c/183720/
>>>>>>> FETCH_HEAD
