% Chapter Template

\chapter{Analysis and Results} % Main chapter title

\label{Chapter6} % Change X to a consecutive number; for referencing this chapter elsewhere, use \ref{ChapterX}

\lhead{Chapter 6. \emph{Analysis and Results}} % Change X to a consecutive number; this is for the header on each page - perhaps a shortened title

\section{Analysis}
The final purpose of our system is to provide analysis on the main issues present on the infrastructure. The way we processes here is quite straightforward.

\begin{enumerate}
\item Putting together the related information
\item Spotting the anomalies
\item Proposing adequate and possible cures
\end{enumerate}

Like any system, we had to define domains on which the tool will be able to work. One of the first steps of our thesis was to determine what were the main issues. It's here that we will use those. Each of these problems will be a domain in which we will define what are the relevant data, how they can help us and how to react after having analysed them. The purpose of this chapter is to present you these domains and the methodology used inside them. Each of them have their particular needs and their adequate treatments.

\section{Wifi}
\subsection{Protocol Used}
\subsection{Rogue Access Points}
\subsection{Load of the Access Point}
\subsection{Perception of the users}
It's here that our probing devices will be useful. As they simulate the connection of a lambda user, we can compute the time and approximate the quality perceived by the users. 

\section{Controller}

\section{DHCP}

\section{Radius}


