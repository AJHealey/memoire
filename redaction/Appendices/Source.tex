% Appendix Template


\chapter{Source Code} % Main appendix title
\label{AppendixA} % Change X to a consecutive letter; for referencing this appendix elsewhere, use \ref{AppendixX}
\lhead{Appendix A. \emph{Source Code}} % Change X to a consecutive letter; this is for the header on each page - perhaps a shortened title

The source code of our monitoring application is available on Github at the following address: \texttt{https://github.com/cwayembergh/netwobserver/}.
Here is the list and descriptions of the contents of this repository.

\begin{itemize}
\item[-] \texttt{analyse/}: Module that performs analysis and aggregations on the data.
\item[-] \texttt{analyse/observation/monitoring.py}: Methods detecting the anomalies.
\item[-] \texttt{analyse/computation/aggregation.py}: Methods performing the aggregations and computations on the data.
\item[-] \texttt{gatherer/}: Module that is responsible to gather the data.
\item[-] \texttt{gatherer/log/}: Module that performs the parsing of the log files.
\item[-] \texttt{gatherer/snmp/}: Abstractions for the SNMP requests.
\item[-] \texttt{gatherer/probe/}: Module that manage the connections with the probes.
\item[-] \texttt{netwObserver/settings.py}: Settings of the application. Contains the user preferences.
\end{itemize}


